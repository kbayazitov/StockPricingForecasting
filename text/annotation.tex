\newpage

\begin{abstract}

Исследуется проблема повышения качества моделей прогнозирования временных рядов на примере динамики курса акций. Рассматривается метод включения в модель внешних данных. Вводится предположение, что агрегированные знания опытных инвесторов повышают качество тестируемой модели. Рассматриваются нейросетевые методы машинного обучения и проводятся вычислительные эксперименты на реальных данных.

\smallskip

\textbf{Ключевые слова}: временные ряды, нейронные сети, краткосрочное прогнозирование

\end{abstract}
