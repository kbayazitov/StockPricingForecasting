\newpage

\begin{abstract}
Исследуется проблема повышения качества прогнозирования временных рядов при использовании дополнительных данных на примере курса акций. Вводится предположение, что агрегированные знания опытных инвесторов повышают качество базовой модели. Рассматриваются нейросетевые методы машинного обучения и проводятся вычислительные эксперименты на реальных данных.

\smallskip

\textbf{Ключевые слова}: временные ряды, нейронные сети, дистилляция знаний

\end{abstract}
