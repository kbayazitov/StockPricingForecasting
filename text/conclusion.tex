\newpage

\section{Заключение}

\begin{table}[h!t]
\begin{center}
\caption{Результаты экспериментов}
\label{final_results}
\resizebox{\linewidth}{!}{
\begin{tabular}{|c|c|c|c|c|}
\hline
\textbf{Выборка} & \textbf{Модель} & \textbf{\begin{tabular}[c]{@{}c@{}}Дополнительные\\ данные\end{tabular}} & \textbf{\begin{tabular}[c]{@{}c@{}}Корреляция\\ Пирсона\end{tabular}} & \textbf{MSE} \\
\hline
\hline
& ARIMA & & $0{,}340$ & $0{,}0183$ \\
\hline
\hline
\multirow{4}{*}{YNDX-Train5} & \multicolumn{1}{|c|}{\multirow{2}{*}{Seq2Seq LSTM}} & --- & $0{,}476 \pm 0{,}027$ & $0{,}0175 \pm 0{,}0003$ \\ \cline{3-3} \cline{4-4} \cline{5-5}
                            & \multicolumn{1}{|c|}{}                                    & Автоследование & $0{,}510 \pm 0{,}036$ & $0{,}0171 \pm 0{,}0001$ \\ 
                            \cline{2-3} \cline{4-4} \cline{5-5}
                            & \multicolumn{1}{|c|}{\multirow{2}{*}{Seq2Seq Transformer}} & --- & $0{,}379 \pm 0{,}039$ & $0{,}0225 \pm 0{,}0005$ \\ 
                            \cline{3-3} \cline{4-4} \cline{5-5}
                            & \multicolumn{1}{|c|}{}                                    & Автоследование & $0{,}401 \pm 0{,}035$ & $0{,}0219 \pm 0{,}0004$ \\ 
\hline
\hline
\multirow{4}{*}{YNDX-Train15} & \multicolumn{1}{|c|}{\multirow{2}{*}{Seq2Seq LSTM}} & --- & $0{,}553 \pm 0{,}004$ & $0{,}0186 \pm 0{,}0008$ \\ \cline{3-3} \cline{4-4} \cline{5-5}
                            & \multicolumn{1}{|c|}{}                                    & Автоследование & $0{,}599 \pm 0{,}006$ & $0{,}0175 \pm 0.0007$ \\ 
                            \cline{2-3} \cline{4-4} \cline{5-5}
                            & \multicolumn{1}{|c|}{\multirow{2}{*}{Seq2Seq Transformer}} & --- & $0{,}415 \pm 0{,}010$ & $0{,}0229 \pm 0{,}0006$ \\ 
                            \cline{3-3} \cline{4-4} \cline{5-5}
                            & \multicolumn{1}{|c|}{}                                    & Автоследование & $0{,}440 \pm 0{,}001$ & $0{,}0218 \pm 0{,}0009$ \\
\hline
\hline
\multirow{4}{*}{YNDX-Train30} & \multicolumn{1}{|c|}{\multirow{2}{*}{Seq2Seq LSTM}} & --- & $0{,}478 \pm 0{,}006$ & $0{,}0195 \pm 0{,}0005$ \\ \cline{3-3} \cline{4-4} \cline{5-5}
                            & \multicolumn{1}{|c|}{}                                    & Автоследование & $0{,}538 \pm 0{,}007$ & $0{,}0186 \pm 0{,}0017$ \\ 
                            \cline{2-3} \cline{4-4} \cline{5-5}
                            & \multicolumn{1}{|c|}{\multirow{2}{*}{Seq2Seq Transformer}} & --- & $0{,}406 \pm 0{,}011$ & $0{,}0211 \pm 0{,}0010$ \\ 
                            \cline{3-3} \cline{4-4} \cline{5-5}
                            & \multicolumn{1}{|c|}{}                                    & Автоследование & $0{,}425 \pm 0{,}009$ & $0{,}0210 \pm 0{,}0009$ \\
\hline
\end{tabular}
}
\end{center}
\end{table}


В работе рассмотрена проблема повышения качества моделей прогнозирования временных рядов на примере динамики курса акций. Рассмотрены методы прогнозирования на основе рекуррентных сетей и моделей трансформеров. Был предложен подход включения в модель дополнительных данных --- агрегированных знаний опытных инвесторов.

В ходе экспериментов, проведенных на реальных данных, было показано, что предложенный метод работает и повышает качество тестируемых моделей. Результаты экспериментов представлены в таблице~\ref{final_results}. 

Из таблицы видно, что качество модели зависит от размера входного окна: модели, обученные на выборке с входным окном размера 15, имеют наилучшее качество. Также во всех экспериментах качество тестируемой модели повышается при использовании дополнительных данных модели автоследования.