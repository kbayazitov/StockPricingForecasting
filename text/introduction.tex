\newpage

\section{Введение}

\paragraph{Актуальность темы.} 
Предсказание цен на акций с помощью моделей машинного обучения является сложной задачей из-за высокой степени шума и множества факторов, влияющих на поведение цен.
В этом случае можно использовать перенос знаний с агрегированных ответов опытных инвесторов - \textit{модели автоследования} на \textit{базовую модель} предсказания.

\paragraph{Цель работы.} Одним из способов повышения качества алгоритма машинного обучения является использование дополнительных данных. Цель данной работы заключается
в повышении качества модели машинного обучения, предсказывающей временные ряды на примере курса акций. Для этого предлагается использовать передачу знаний от опытных инвесторов.

\paragraph{Новизна.}
Предложен подход, основанный на предположении о том, что внешние факторы, влияющие на курс акций, заложены в ответы опытных инвесторов.

\subsection{Обзор предметной области}

\begin{definition}
Модель автоследования --- агрегированные ответы опытных инвесторов о решении продажи или покупки акций, которые используются при обучении базовой модели.
\end{definition}

\begin{definition}
Базовая модель --- модель, при обучении которой используются ответы модели автоследования.
\end{definition}

\subsection{Предложенный метод}

Предлагается при обучении базовой модели использовать помимо данных временного ряда также ответы модели автоследования. Ожидается, что качество новых моделей будет превышать качество моделей, в обучении которых не использовались ответы модели автоследования.

В качестве экспериментальных данных используются реальные данные курса акций YNDX и ответы опытных инвесторов --- пользователей Тинькофф Инвестиций.