\newpage

\section{Введение}

\paragraph{Актуальность темы.} 
Прогнозирование цен на акции с помощью моделей машинного обучения является сложной задачей из-за высокой степени шума и множества факторов, влияющих на поведение цен.
В этом случае помимо данных временного ряда можно использовать агрегированные знания опытных инвесторов.

\paragraph{Цель работы.} Одним из способов повышения качества алгоритма машинного обучения является обогащение выборки - использование дополнительных данных. Цель данной работы заключается в повышении качества нейросетевых моделей прогнозирования временных рядов на примере динамики курса акций. Для этого предлагается использовать знания опытных инвесторов.

\paragraph{Новизна.}
Предложен подход, основанный на предположении о том, что внешние факторы, влияющие на курс акций, заложены в ответы опытных инвесторов.

\begin{definition}
Модель автоследования --- временной ряд, составленный из агрегированных ответов опытных инвесторов о решении продажи или покупки акций.
\end{definition}

\begin{comment}
\begin{definition}
Тестируемая модель --- модель, при обучении которой используются ответы модели автоследования.
\end{definition}
\end{comment}

\subsection{Обзор предметной области}

Основным подходом краткосрочного прогнозирования временных рядов является использование моделей семейства ARIMA, описанных в работе~\cite{ARIMA}. Данные модели основаны на использовании авторегрессии и скользящего среднего. В работах~\cite{Критерии стационарности 1, Критерии стационарности 2} рассматриваются тесты стационарности, являющейся важным условием для работы с моделями из данного семейства. Задача прогнозирования цен на акции с использованием модели ARIMA описана в~\cite{Stock price ARIMA}.

Нейросетевые методы прогнозирования используют Кодировщик-Декодировщик архитектуру, описанную в~\cite{EncoderDecoder}. Кодировщик преобразует входную последовательность в векторное представление, используемое Декодировщиком при построении выходной последовательности, однако в качестве Кодировщика и Декодировщика также могут выступать различные архитектуры. В работе~\cite{EncoderDecoder} в качестве Кодировщика и Декодировщика используется архитектура рекуррентной нейронной сети LSTM~\cite{LSTM1, LSTM2}.

В работе~\cite{RNN Attention} предлагается использовать помимо архитектуры Кодировщика и Декодировщика также механизм внимания - меру сходства состояний модели. Различные виды данного механизма описаны в~\cite{Attention Mechanisms}. А в работе~\cite{Self-Attention} описан механизм самовнимания - метод создания векторных представлений последовательностей.

Современные методы прогнозирования основаны на применении моделей трансформеров, описанных в~\cite{Attention is all you need}. Данный подход основан на использовании механизма внимания без применения рекуррентности. Еще одним важным аспектом данных моделей является позиционное кодирование~\cite{Positional Encoding} - включение дополнительной информации о порядке передаваемой последовательности.

Последние исследования в области прогнозирования временных рядов используют архитектуры эффективных трансформеров - модификаций базовой архитектуры трансформеров, направленных на повышение их производительности, снижение вычислительных затрат и улучшение обработки последовательностей. В работе~\cite{Informer} описана архитектура Informer, использующая механизм разреженного внимания - уменьшение размерности вектора запроса. Архитектура Autoformer~\cite{Autoformer} использует слой декомпозиции - разделение временных рядов на сезонные и трендовые компоненты и автокорреляцию в качестве механизма внимания.

Задача прогнозирования временных рядов находит свое применение во многих областях. Так, в~\cite{iTransformer} в качестве экспериментальных данных используются такие датасеты, как курс валют, метеорологические данные, данные о потреблении электричества и другие данные реального мира.

\subsection{Предложенный метод}

Предлагается в тестируемой модели использовать помимо данных временного ряда также ответы модели автоследования. Ожидается, что качество полученных моделей будет превышать качество моделей без использования ответов модели автоследования.

В качестве экспериментальных данных используются реальные данные подневной динамики курса акций YNDX и ответы опытных инвесторов - авторов стратегий автоследования Тинькофф Инвестиций.